\section{UVOD}
\subsection{Pozadina problema}
\justifying{
U poslednjih nekoliko godina, generativni AI modeli doživeli su izuzetan tehnološki napredak. Ključni trenutak dogodio se razvojem OpenAI GPT serije, posebno modela ChatGPT, koji omogućava generisanje koherentnog teksta u obliku ljudske konverzacije. Paralelno sa tim, GitHub Copilot, baziran na Codex modelu, doneo je revoluciju u načinu na koji programeri pišu i razumeju kod.\\
Osim ova dva alata, značajan uticaj ima i Google Gemini (ranije Bard), ali s obzirom na fokus ovog istraživanja, teza će biti ograničena na ChatGPT i Copilot, kako bi se omogućilo detaljno poređenje njihovih primena i izazova koje donose.
Primena ovih alata proširila se van okvira tradicionalne upotrebe u istraživanju i razvoju, te se danas koriste u:
Obrazovanju: pisanje eseja, rešavanje zadataka, interaktivno učenje.\\
Programiranju: automatsko generisanje koda, refaktorisanje, dokumentacija.
Pisanoj komunikaciji: kreiranje izveštaja, sažetaka, poslovnih pisama.
Administraciji: automatizacija obrada zahteva i interakcija sa korisnicima.
Ovaj fenomen donosi pitanje: Da li su ovi alati postali neophodna asistivna tehnologija, ili alat za zaobilaženje ljudskog truda i odgovornosti?
}

\subsection{  i doprinos istraživanja}
\justifying
Glavni cilj ovog istraživanja je analiza realnih scenarija upotrebe i zloupotrebe generativnih AI alata na primeru ChatGPT-a i Copilot-a, uz identifikaciju njihovih etičkih, pravnih i tehničkih implikacija. \\
Specifični ciljevi uključuju:
{\small
\begin{itemize}
    \item Prikaz pozitivnih primera upotrebe, kao što su asistencija u kodiranju, automatizacija pisanih zadataka, i unapređenje edukativnog procesa.
    \item Identifikacija zloupotreba, uključujući pisanje seminarskih radova bez citiranja, generisanje malicioznog koda, i kreiranje dezinformacija.
    \item Analizu pravnih izazova, sa fokusom na autorska prava i odgovornost za AI-generisan sadržaj.
    \item Sagledavanje etičkih dilema, koje se tiču vrednosnog usklađivanja AI sistema sa društvenim normama.
    \item Ispitivanje tehničkih rizika, kao što su sigurnosne ranjivosti u kodu generisanom Copilot-om ili mogućnosti "prompt injection" napada na ChatGPT.
\end{itemize}
}
\par\noindent
Doprinos rada ogleda se u sistematičnoj analizi kako AI alati transformišu profesionalne i obrazovne procese, ali i u preporukama za odgovornu primenu ovih tehnologija, sa ciljem očuvanja integriteta i sigurnosti.

\subsection{Struktura rada}
\justifying
Struktura rada organizovana je u osam poglavlja, od uvodnih razmatranja do analiza konkretnih slučajeva i budućih perspektiva:

\begin{itemize}
    \item \textbf{main.tex} — Glavni \LaTeX{} fajl koji uključuje sva poglavlja i postavke dokumenta.
    \item \textbf{bibliography.bib} — Datoteka sa referencama i literaturom.
    \item \textbf{images/} — Direktorijum sa slikama i ilustracijama korišćenim u radu.
    \item \textbf{Poglavlja/} — Direktorijum sa pojedinačnim fajlovima za svako poglavlje:
    \begin{itemize}
        \item \texttt{naslovna\_strana.tex} — Naslovna strana rada.
        \item \texttt{apstrakt.tex} — Apstrakt rada.
        \item \texttt{uvod.tex} — Uvod i pozadina problema.
        \item \texttt{teorijski\_okvir.tex} — Teorijski okvir generativne veštačke inteligencije.
        \item \texttt{pravni\_okvir.tex} — Pravni i regulatorni okvir.
        \item \texttt{upotreba.tex} — Primeri pozitivne upotrebe alata.
        \item \texttt{zloupotreba.tex} — Analiza zloupotreba i rizika.
        \item \texttt{studije\_slucaja.tex} — Detaljne studije slučaja.
        \item \texttt{preporuke.tex} — Preporuke za odgovornu primenu.
        \item \texttt{buducnost.tex} — Perspektive budućeg razvoja i etička pitanja.
    \end{itemize}
\end{itemize}

\section{GENERATIVE AI - TEORIJSKI OKVIR}

\subsection{Evolucija i Uspon Generativne Veštačke Inteligencije}
\justifying
Pojava ChatGPT-a 30. novembra 2022. godine označila je ključni trenutak u popularizaciji generativne veštačke inteligencije (GAI) među širom javnosti. Ovo dostignuće ne može se posmatrati izolovano, već kao rezultat višedecenijskog razvoja oblasti veštačke inteligencije, koja je svoje začetke imala još 1956. godine na letnjem projektu na Dartmouth koledžu, koji je predvodio Džon Makarfi. Tada je prvi put postavljena ideja o stvaranju mašina sposobnih da izvršavaju zadatke koji zahtevaju ljudsku inteligenciju, poput obrade prirodnog jezika, računarskog vida, robotike i drugih.

\subsubsection{Razvoj klasičnih algoritama mašinskog učenja}
Objavljivanje ChatGPT-a 30. novembra 2022. pokrenulo je eksponencijalni porast revolucionarne i široko rasprostranjene popularnosti GAI među širom javnošću. Ovo izuzetno dosti-\\gnuce  može se pratiti do letnjeg projekta iz 1956. godine na koledžu Dartmut, koji je predvodio Makarti; obeležavajući početak veštačke inteligencije (VI)~\cite{ref3}. Cilj poduhvata bio je razvoj mašina sa sposobnošću da obavljaju zadatke koji tipično zahtevaju ljudsku inteligenciju~\cite{ref4,ref5,ref6,ref7,ref8}. Ovi zadaci uključuju računarski vid, obradu prirodnog jezika (NLP)~\cite{ref9,ref10}, robotiku i mnoge druge. Od tada je postignut značajan napredak u osposobljavanju mašina sposobnošću da govore, hodaju, misle i deluju kao ljudi. Primetno je da se pojavio niz algoritama.

\addto\captionsenglish{\renewcommand{\tablename}{Tabela}}

\addto\captionsenglish{\renewcommand{\tablename}{Tabela}}
\arrayrulecolor{apstrakt_color}

\begin{table}[h!]
\centering
\renewcommand{\arraystretch}{1.4}
\setlength{\tabcolsep}{8pt}
\begin{tabularx}{\textwidth}{>{\bfseries}l X}
\rowcolor{apstrakt_color!30}
\textbf{Algoritam} & \textbf{Opis} \\
\hline
Regresija & Osnovni statistički model koji predviđa kontinualne vrednosti na osnovu ulaznih podataka. \\
\hline
Perceptron & Jednostavan algoritam za binarnu klasifikaciju, temelj za razvoj neuronskih mreža. \\
\hline
Stablo odluke & Hijerarhijska struktura koja klasifikuje podatke kroz niz uslova. \\
\hline
K najbližih suseda (K-NN) & Neparametarski algoritam koji klasifikuje podatke na osnovu najbližih primera u prostoru karakteristika. \\
\hline
Naivni Bajes & Verovatnosni klasifikator baziran na teoremi verovatnoće, efikasan i kod ograničenih resursa. \\
\hline
Backpropagation & Osnovni mehanizam za treniranje višeslojnih neuronskih mreža korišćenjem gradijentnog spuštanja. \\
\hline
SVM (Support Vector Machines) & Algoritam koji pronalazi optimalnu granicu klasifikacije u prostoru podataka. \\
\hline
Random Forest & Ansambl model baziran na više stabala odluke, poznat po robusnosti i preciznosti. \\
\hline
\end{tabularx}
\caption{Pregled klasičnih algoritama mašinskog učenja}
\end{table}

Ovi algoritmi čine okosnicu klasičnog mašinskog učenja i predstavljaju temelj za kasniji razvoj kompleksnijih modela.
\newpage
\subsubsection{Uspon dubokog učenja}
Sa porastom računarskih resursa i dostupnošću velikih količina podataka, otvorila su se vrata za duboko učenje. Ključne arhitekture koje su oblikovale ovu fazu uključuju:
\newpage 